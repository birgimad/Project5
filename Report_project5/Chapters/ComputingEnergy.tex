\section{Computing the Energy}
\label{sec:ComputingEnergy}
In order to test whether the energy is conserved, the initial energy of the system can be calculated and printed together with the final energy after a specific time interval.
According to the conservation of energy, these are equal. 

The total energy $E_{tot}$ of the system is found by summing up the potential energy $E_{pot}$ and kinetic energy $E_{kin}$ of the $N$ bodies that constitutes the system.
The total potential energy is calculated as 
\begin{align}
	E_{pot} = \sum _{1=0} ^N \sum _{j \neq i} \frac{m_i m_j}{r_{ij}}
\end{align}
in which $m_i$ and $m_j$ are the masses of the $i$'th and $j$'th body, respectively, $r_{ij} = | \v{r}_i - \v{r}_j |$ is the distance between the two bodies, and $G$ is the gravitational constant. 
The total kinetic energy of the system is calculated as 
\begin{align}
	E_{kin} = \frac{1}{2} \sum _{1=0} ^N m_i v_i ^2
\end{align}
with $v_i$ being the speed of the $i$'th particle calculated as 
$v_i = \sqrt{v_{ix}^2 +v_{iy}^2 +v_{iz}^2}$, and $m_i$ is the corresponding mass of that body. 

The c++ code for computing the total energy of the system is given here below.
When the kinetic energy is calculated, $v_i$ is not explicitly calculated. Instead $v_i^2$ is calculated to reduce the number of floating point operations. 
\begin{lstlisting}
for (int i=0; i<number_of_particles; i++){
	for (int k=0; k<3; k++){
    	kin_en(i) += v(i,k)*v(i,k);
    }
    kin_en(i) = 0.5*m(i)*kin_en(i);
    for (int j=0; j<number_of_particles; j++){
        if (j != i){
        	pot_en(i) += pow(distance_between_particles(i,j),-1.0)*m(j);
        }
    }
    pot_en(i) = pot_en(i)*G*m(i);
    tot_en(i) = kin_en(i)+pot_en(i);
}
\end{lstlisting}
