\section{Newtonian two-body problem in three dimension}
\label{Newton2body3D}
\fxnote{make short intro, e.q. drawing}
$\v{r}(t)$ is the three-dimensional space vector consisting of the coordinated $(x(t),y(t),z(t))$, whilst $\v{v}(t)$ is the three-dimensional velocity vector with coordinates $(v_x(t),v_y(t),v_z(t))$, both of which are dependent on time. 

In general, the differential equation that is considered is
\begin{align}
	\frac{dy}{dt} = f(t,y)
	\label{eq:diffEq1}
\end{align}
Which yields that
\begin{align}
	y(t) = \int f(t,y) dt
\end{align}
\fxnote{do we need to write $y_{i+1}$ eq from p. 250 in lecture notes??}
For the two bodies in a three dimensional Newtonian gravitational field this corresponds to six coupled differential equations given by the vector equations
\begin{align}
	\frac{d\v{r}}{dt} = \v{v}
	\qquad \text{and} \qquad
	\frac{d\v{v}}{dt} = - \frac{G M_1 M_2}{r^3} \v{r}
	\label{eq:diffEq2}
\end{align}
\fxnote{maybe we should divide by mass as on p. 248??}
in which $M_1$ and $M_2$ \fxnote{fix the this with $M_1$ and $M_2$} are the masses of the two bodies, respectively, whilst $r$ is the distance between the bodies.
The equations in \eqref{eq:diffEq2} are computed by the script given below in which $drdt$ corresponds to the derivative of the coordinates of the position, and $dvdt$ corresponds to the derivative of the velocity coordinates. 
\begin{lstlisting}
void Derivative(double r[3], double v[3], double (&drdt)[3], double (&dvdt)[3], double G, double mass){
    drdt[0] = v[0];
    drdt[1] = v[1];
    drdt[2] = v[2];

    double distance_squared = r[0]*r[0] + r[1]*r[1] + r[2]*r[2];
    double newtonian_force = -G*mass/pow(distance_squared,1.5);
    dvdt[0] = newtonian_force*r[0];
    dvdt[1] = newtonian_force*r[1];
    dvdt[2] = newtonian_force*r[2];
}
\end{lstlisting}
