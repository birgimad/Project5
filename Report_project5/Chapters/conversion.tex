\section{Transformation between units}
\label{sec:Conversion}
When considering at a planetary system or a larger system like a galaxy it is inconvenient to use the SI units for length and time.
Instead, to investigate the evolution of astronomical systems, it is an advantage to use days, years (yr) or even longer time periods as the unit of time, and astronomical units (AU) or light-years (ly) as the unit of distance.
The change in unit system, evidently changes the considered constant, the gravitational constant $G$, which in SI units is given as
\begin{align*}
	G = 6.67\cdot 10^{-11} \frac{\textrm{Nm}^2}{\textrm{kg}^2}
\end{align*}
For a planetary system like the Earth-Sun system it is better to consider distances in AU instead of meters, and use days as a measure of time, as the planet doesn't move far on its orbit in a second.
Furthermore it's an advantage to express the masses in the system in units of solar mass. 
Hence the constants have to be transformed into these unit systems. 

The gravitational constant G is transformed using
\begin{align*}
	1 \textrm{ AU} = 1.495\cdot10^{11} \textrm{ m}
	\qquad \text{and} \qquad
	1 \textrm{ M}_{\odot} = 1.989 \cdot 10^{30} \textrm{ kg}
\end{align*}
giving the gravitational constant for the planetary system
\begin{align*}
	G = 2.96\cdot 10^{-4} \frac{\textrm{AU}^3}{\textrm{days}^2 \textrm{M}_{\odot}}
\end{align*}
which is convenient when considering a planetary system.
 
For a star cluster the distances are greater and the time scales are larger than for the planetary system.
Therefore it's more convenient to use years as the unit of time and light-years (ly) as the unit of distance. 
\begin{align*}
	1 \textrm{ yr} = 3.1536\cdot10^7\textrm{s}
	\qquad \text{and} \qquad
	c = 2.008\cdot 10^8 \frac{\textrm{m}}{\textrm{s}}
\end{align*}
in which $c$ is the speed of light.
This yields that 1 ly is
\begin{align*}
	1 \textrm{ ly} = 9.45 \cdot 10^{15} \textrm{ m}
\end{align*}

Giving the gravitational constant for stellar distances 
\begin{align*}
	G = 1.536\cdot 10^{-13} \frac{\textrm{ly}^3}{\textrm{yr}^2 {\textrm{M}}_{\odot}}
\end{align*}



Finding the gravitational constant in units of $\tau:{chrunch}$. 
\begin{align*}
G = 986.96 \frac{\textrm{ly}^3}{\tau_{crunch}^2 \textrm{m}_{mean}}
\end{align*}

