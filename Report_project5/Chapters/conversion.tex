
\subsection{Transformation between units}

When looking at a planetary system or a larger system like a galaxy it is inconvenient to use the standard units like meters and seconds. So to make it easier to look at how thing are evolving it is better to use days, years, astronomical units or light years.

The gravitational constant in standard units


$ G = 6.67*10^{-11} \frac{\textrm{Nm^2}}{\textrm{kg^2}}$ 
 

For a planetary system like the Earth sun system it is better to look at distance in Astronomical units instead of meters, and use days as a measure of time, as the planet doesn't move much in a second and it would give a to precise measurement also the mass is easier to use if it's in units of solar mass. So to avoid this problem the constants have to be transformed into these unit systems. 

The gravitational constant G is transformed using

$1 AU = 1.495*10^{11} m$ 

$1 M_{sun} = 1.989 * 10 {30} kg$

Then converting the gravitational constant so it can be used in the planetary system

$ G = 2.96* 10^{-4} \frac{\textrm{AU^2}}{days^2m_sun}$ 


So for the star cluster the distances are so great and the time it takes for the stars to move around is quite long, so it is better to use years for time and light years for the distance. 

$Year (yr) = 3.1536*10^7s$

$Light speed (c) = 2.008*10^8 \frac{\textrm{m}}{\textrm{s}}$

$Light year (ly)= 9.45 * 10^15 \textrm{m} $ 

Giving the gravitational constant

$G = 1.536* 10^{-13} \frac{\textrm{ly^2}}{yr^2m_sun}$

