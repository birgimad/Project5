\subsection{Velocity-Verlet method}
\label{sec:methodVV}
The basic idea of the Velocity-Verlet algorithm is to write the Taylor expansion of the position in Newton’s equation, one forward step and one backward step in time with step length $\delta t$ as
\begin{align}
	\v{r}(t_i \pm \delta t) = \v{r}(t_i) \pm \v{v} (t_i) \delta t
	+ \v{a} (t_i) \frac{\delta t ^2}{2}  \pm \frac{\delta t ^3}{6} \frac{d^3 \v{r}(t_i)}{dt^3} + \mathcal{O}(\delta t ^4 )
	\label{eq:TaylorExpVVmethod}
\end{align}
in which $\v{v}(t_i) = d\v{r}(t_i)/dt$ is the velocity, and $\v{a}(t_i) = d^2 \v{r}(t_i)/dt^2$ is the acceleration at time $t_i$.
Adding the two expressions in \matref{eq:TaylorExpVVmethod} gives
\begin{align}
	\v{r} (t_i +\delta t) = 2\v{r} (t_i) - \v{r} (t_i -\delta t)  + \v{a} (t_i ) \delta t^2 + \mathcal{O} (\delta t ^4 )
	\label{eq:TaylorExpVVmethod2}
\end{align}
which has a truncation error that goes as $\mathcal{O} (\delta t ^4 )$.
Now, using $\v{r}(t_i-\delta t) = \v{r}(t_i) - \v{v}(t_i) \delta t + \v{a} \delta t^2 /2 + \mathcal{O} (\delta t^3)$, yield that the position at time $t_i+\delta t$ can be determined as 
\begin{align}
	\v{r}(t_i+\delta t) = \v{r}(t_i) + \v{v} (t_i) \delta t + \frac{1}{2} \v{a} (t_i) \delta t^2 
	\label{eq:VVmethodNextPosition}
\end{align}
Since the velocity is not included in \matref{eq:TaylorExpVVmethod2}, it is computed through the Velocity-Verlet scheme where position, velocity and acceleration at time $t_i+\delta t$ is computed from the Taylor expansion as
\begin{align}
	\v{v} (t + \delta t) = \v{v} (t) + \frac{1}{2} ( \v{a} (t) + \v{a} (t + \delta t) ) \delta t
	\label{eq:VVmethodNextVelocity}
\end{align}
The velocity at time $t_i+\delta t$ is in the algorithm computed by first calculating 
\begin{align}
	\v{v}_{part1} (t + \delta t) = \v{v} (t) + \frac{1}{2}  \v{a} (t)  \delta t
	\label{eq:VVmethodNextVelocityPart1}
\end{align}
and then use the \textit{Derivative} function to determine $\v{a} (t + \delta t)$, which is then used to compute the remaining term of \matref{eq:VVmethodNextVelocity} as 
\begin{align}
	\v{v}_{part2} (t + \delta t) = \frac{1}{2} \v{a} (t + \delta t) \delta t
	\label{eq:VVmethodNextVelocityPart2}
\end{align}

The velocity-Verlet method uses the algortihm \textit{Derivative} described in \secref{Newton2body3D}, to generate the six differential equations, in the following while-loop that runs until reaching the final time in time steps of length $\delta t = (t_{initial} - t_{final})/(\# time steps)$.
\begin{lstlisting}
    while(time<=t_final){
    Derivative(r,v,drdt,dvdt,G,mass);
    for(int i=0; i<6 ; i++){
    r[i] = r[i]+dt*drdt[i] + 0.5 * dt * dt * dvdt[i];
    v_partly[i] = drdt[i] + 0.5 * dt * dvdt[i];
    dvdt[i] = v_partly[i];
    }
    Derivative(r,v,drdt,dvdt,G,mass);
    for(int i=0; i<n ; i++){
    v[i] = v_partly[i] + 0.5 * dt * dvdt[i];
    }
    time += dt;
    }
\end{lstlisting}