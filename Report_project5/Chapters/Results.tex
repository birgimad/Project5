\chapter{Results and Discussion}
\label{chap:Results}
The starting point is solving the two body system, where Earth- and Sun-like forms the two bodies with masses $1\textrm{M}_{\odot}$ and $3\cdot 10^{-6} \textrm{M}_{\odot}$, respectively, for which $\textrm{M}_{\odot}$ is the solar mass. 
With the help of the Runge-Kutta method and the Velocity-Verlet method introduced in \secref{Newton2body3D}, the problem is solved both with a stationary Sun-like body relative to the frame of reference, and with both Earth and Sun moving relative to the coordinate system, with an initial velocity (0,0,0) and initial position an (1,1,1) for the Sun. 
For earth the initial position is assigned to be (2,1,1) and initial velocity (0,0.017,0). 

For an $N$ body system, the movement of the bodies with the evolution of time is estimated with the Velocity-Verlet method. 
From the result analysis the behaviour of the system is unfold.
\fxnote{ok, is this what we want to do?}

The results from running the codes described in \chapref{chap:method} for computing the blah blah blah ?? can be found in the GitHub folder  \url{https:/??}, together with the MatLab scripts for the plots presented in this chapter. \fxnote{fix these lines}
