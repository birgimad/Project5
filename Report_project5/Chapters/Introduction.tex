\chapter{Introduction}
This project focuses on numerical simulation of an open cluster using Newtonian gravity. Open clusters contains a few thousand gravitationally bound stars formed as a result of collapse of a molecular cloud. Many of the variable parameters among these stars are constant as they have the same age and same composition and this make them vital in the study of stellar evolution. 

Here two of the finite different methods for solving differential equations namely the fourth order Runge- Kutta method and the Velocity Verlet method are used to develop the code for numerical simulation. At first the stability of these two methods are tested which is important when studying the statistical properties of a system with a large number of particles. This is done by implementing Newtonian two-body problem in three dimension with the two methods and comparing the output from both of them for larger time steps, longer time steps and the time used to advance one time step. Further the system is extended to a cluster containing N particles which is confined inside a sphere of radius in the order of light years. Initially the particles are made to be at rest and using appropriate C++ functions random numbers corresponding to particles masses and positions are generated in such a way that the masses are randomly distributed by a Gaussian distribution around ten solar masses with a standard deviation of one solar mass and the particles are uniformly distributed in the Cartesian coordinates x,y and z within the sphere. 

An estimation of the stability of both the Fourth order Runge Kutta and Velocity Verlet method is made by running the respective codes with gravitational constant G in units of $\tau_{crunch}$, the time period at which the system collapses to a singularity when number of particles tends to infinity and studying the initial and final distribution of particles in the radial direction for time period $\tau_{crunch}$ with different step lengths. The time taken for the system to reach equilibrium is studied for different $\tau_{crunch}$. An analysis of the energy conservation is made by calculating the kinetic and potential energy of the system. 
 


The main part of the report consists of two chapters: The methods and theory chapter, and the results and discussion chapter.