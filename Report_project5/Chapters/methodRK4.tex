\subsection{Fourth Order Runge-Kutta Method}
\label{sec:methodRK4}
- Remember to write about accuracy of algorithm!!
\fxnote{fix this}

The Runge-Kutta method is based on Taylor expansions, with the next function value after a times step $\delta t = t_i - t_{i+1}$ being computed from four more or less improved slopes of the function in the points $t_i$, $t_i + \delta t /2$ and $t_{i+1}$.
  
The first step of the RK4 method is to compute the slope $k_1$ of the function in $t_i$ by
\begin{align*}
	k_1 = \delta t f(t_i , y_i)
\end{align*}
Then the slope $k_1$ at the midpoint is computed from $k_1$ as
\begin{align*}
	k_2 = \delta t f(t_i + \delta t /2, y_i + k_1 /2 )
\end{align*}
The slope at the midpoint is then improved from $k_2$ by
\begin{align*}
	k_3 = \delta t f(t_i + \delta t /2, y_i + k_2 /2 )
\end{align*}
from which the slope $k_4$ at the next step $y_{i+1}$ is predicted to be
\begin{align*}
	k_4 = \delta t f(t_i + \delta t, y_i +k_3 )
\end{align*}
From the computed slopes $k_1$, $k_2$, $k_3$ and $k_4$, the function value at $t_i + \delta t$ is computed as
\begin{align}
 y_{i+1} = y_i + \frac{1}{6} (k_1 + 2k_2 + 2k_3 +k_4 )
 \label{eq:RK4nextxtep1}
\end{align}
When implementing this for the two-body problem in three dimensions, it boils down to a continuous call of two functions, namely the function 
\textit{Derivative} given in 
\secref{Newton2body3D} and the function 
\textit{updating\_dummies} given below.
\begin{lstlisting}
void updating_dummies(double dt, double drdt[3], double dvdt[3], double (&r_dummy)[3], double (&v_dummy)[3], double number, double (&kr)[3], double (&kv)[3], double r[3], double v[3])
{
    for (int i = 0; i<3; i++){
        kr[i] = dt * drdt[i];
        kv[i] = dt * dvdt[i];
        r_dummy[i] = r[i] + kr[i]/number;
        v_dummy[i] = v[i] + kv[i]/number;
    }
}
\end{lstlisting}
The function \textit{updating\_dummies} computes the values of $k_1$, $k_2$, $k_3$ and $k_4$ for all three space coordinates and velocity coordinates from the derivatives $drdt$ and $dvdt$ computed by the \textit{Derivative} function. 
To compute the next step given by \matref{eq:RK4nextxtep1}, the following succession of function calls are made until the time reaches the final time $t_{final}$ after $(t_{final} - t_{inital}) / \delta t$ time steps.
\begin{lstlisting}
while(time<=t_final){
    Derivative(r,v,drdt,dvdt,G,mass);
    updating_dummies(dt,drdt,dvdt,r_dummy,v_dummy,2,k1r,k1v,r,v);
    Derivative(r_dummy,v_dummy,drdt,dvdt,G,mass);
    updating_dummies(dt,drdt,dvdt,r_dummy,v_dummy,2,k2r,k2v,r,v);
    Derivative(r_dummy,v_dummy,drdt,dvdt,G,mass);
    updating_dummies(dt,drdt,dvdt,r_dummy,v_dummy,1,k3r,k3v,r,v);
    Derivative(r_dummy,v_dummy,drdt,dvdt,G,mass);
    for (int i = 0; i<n; i++){
        k4r[i] = dt*drdt[i];
        k4v[i] = dt*dvdt[i];
    }
    for (int i=0; i<n;i++){
        r[i] = r[i] +(1.0/6.0)*(k1r[i]+2*k2r[i]+2*k3r[i]+k4r[i]);
        v[i] = v[i] +(1.0/6.0)*(k1v[i]+2*k2v[i]+2*k3v[i]+k4v[i]);
    }
    time += dt;
    }
\end{lstlisting}
When including the movement of both bodies relative to the reference system or adding more bodies to the system, $\v{r}$'s, $\v{v}$'s, $\v{k}$'s etc. must be generated for all of the particles, yielding introduction of a for loop over all particles.  